\section{Introduction}

Linked Data\cite{bizer2009linked} is a way to construct a global data space, the Web of Data, by interconnecting many structured data sources within the Linking Open Data\footnote{\url{http://linkeddata.org/}} (LOD) project. These data sources are published under the Resource Description Framework\footnote{\url{http://www.w3.org/RDF/}}(RDF). Each of them may contain millions of RDF triples.

The main idea of Linked Data is to construct typed links between different data sources. Such links describe the relationships between things so that users can browse data among sources by navigating along the links and agents can provide expressive query capabilities over the data on the web just like a local database. An important link is \texttt{owl:sameAs}\footnote{\url{http://www.w3.org/TR/2004/REC-owl-semantics-20040210/\#owl\_sameAs/}} which indicates that the two instances it links refer to the same real-world object. Different data sources may have different emphases in describing things. Various descriptions can be aggregated according to the \texttt{owl:sameAs} links.

Currently, there are more than 300 data sources in LOD while there were only twelve of them in 2007 when the project started. As more and more data sources emerge, there is an urgent demand to provide \texttt{owl:sameAs} links from the new data sources to the existing ones. At the same time, the existing links in LOD are not as extensive as one would hope\cite{hu2011self}. Instance matching is a practical idea for constructing such links. In general, the existing approaches for matching instances in LOD can be divided into two types. One is based on rules and the other is based on the similarity metric of instance pairs. Most of these methods do not always work well in the LOD scenario since they depend on the result of property matchings. Property matching links the properties from different data sources which have similar semantics such as \texttt{foaf:name} and \texttt{dc:title}. Property matching is not trivial since the data sources usually design their own ontologies to describe things. Furthermore, we noticed that although some properties in heterogeneous ontologies can not match, they have some connotative relationships. Their values may be worth considering for instance matching. For example, Freebase\footnote{\url{http://www.freebase.com/}} says that the \texttt{fb:profession} of Steve Jobs is \textquotedblleft Chief Executive Officer\textquotedblright  and his \texttt{fb:place\_of\_death} is \textquotedblleft Palo Alto\textquotedblright, whereas DBpedia\footnote{\url{http://www.dbpedia.org/}} says his \texttt{dbp:residence} is \textquotedblleft Palo Alto California\textquotedblright  and the information about \textquotedblleft Chief Executive Officer\textquotedblright  is in the text of the \texttt{dbp:abstract}. Such information will be ignored in property matching based methods, although it could be significant for human beings to judge whether the two Jobses match. We are inspired to explore the \textquotedblleft common-sense \textquotedblright  used for matching instances. The goal of this paper is to develop an automated instance matching method that is \textquotedblleft common\textquotedblright  and provides high accuracy. Such method should be independent of property matching to achieve \textquotedblleft commonality\textquotedblright .

In this paper, we employ machine learning models for instance matching based on some similarity metrics of instances. The matching instance pairs may have some common features in the similarity metrics of each pair. For example, two matching instances may share some significant words such as \textquotedblleft Palo Alto\textquotedblright  and \textquotedblleft Chief Executive Officer\textquotedblright  which we mentioned above, while the non-matching ones may not. Sharing some significant words is a common feature of the matching instance pairs here. We design a similarity vector independent of property matching to represent such features. Based on this vector, we train a learning model to classify the instance pairs as matching or non-matching. To minimize the demand for training data and promote the performance, we try to use existing instance matching information in LOD for help. A transfer learning method is applied to implement this idea.

We tried our approach on real LOD data sources which were chosen for IM@OAEI2010\footnote{\url{http://oaei.ontologymatching.org/2010/im/index.html}}. Our performance is better than the participating teams'. Another comparative experiment shows that the existing matching information can really help matching instances from the new data source pairs.

The following technical contributions are made:
\begin{itemize}
\item We utilize the values of non-matching properties for instance matching. Such values can be useful but are usually ignored by the existing instance matching approaches.
\item We propose a novel approach for instance matching which is independent of property matching with high accuracy.
\item We use existing \texttt{owl:sameAs} links to help match instances. Due to the heterogeneous ontologies constructed by various data sources, such information is hardly utilized in the existing instance matching approaches.
\end{itemize}

The remainder of this paper is structured as follows. Section \ref{sec:definition&framework} gives some definitions about instance matching and an overview of our proposed approach. Section \ref{sec:extraction} describes the feature extraction process. The selection of machine learning models is discussed in Section \ref{sec:learning}. Experimental results on the LOD datasets from LOD are reported in Section \ref{sec:experiments}. Some related work is discussed in Section \ref{sec:related}. Finally, Section \ref{sec:conclusion} concludes this paper and discusses future work. 