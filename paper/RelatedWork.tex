\section{Related Work}
\label{sec:related}

Although the problem of instance matching has emerged along with the development of LOD in recent years, a similar problem, \textit{Record linkage}, was examined much earlier. Thus a lot of relevant approaches have been proposed.

\subsection{Record Linkage}

\textit{Record linkage}, also known as \textit{duplicate detection} or \textit{object identification}, is a classical problem in the area of databases. The goal of record linkage is to determine the pairs of records that are associated with the same entity across various data sets. It is similar to the instance matching problem. For more than five decades, the traditional database community has discussed this problem a lot. Some surveys can be found in \cite{winkler1999state}, \cite{winkler2006overview} and \cite{elmagarmid2007duplicate}.

The early approaches focus on similarity metrics for single field (column) matching. Many of them are even widely used today, such as edit distance, \textit{Q}-gram distance, etc. The similarity vector in our work is based on $\mathrm{TF}\cdot\mathrm{IDF}$\cite{cohen1998integration} and its variants.

The approaches for multi-field matching are mainly based on probabilistic models, developed by the machine learning community. The learning methods were applied on record linkage by extracting the feature vector of similarities from the comparable fields. Such approaches classify the record pairs as matching or not, employing CART\cite{cochinwala2001efficient}, SVM\cite{bilenko2003adaptive} and so on. Semi-supervised, active and unsupervised learning algorithms have been proposed to reduce the demand for training data. Unlike the properties of linked data sources, the number of fields in a record linkage problem is usually quite small. Thus the comparable fields can be easily found manually. So the data base community doesn't pay attention to the problem known as \textit{Structural heterogeneity}. A simple schema-independent method has also been proposed which treats the whole record as one large field. But experiments in \cite{bilenko2003adaptive} show that SVM based on the similarity metric of multiple comparable fields usually outperforms it. It's easy to see that such an elaborate similarity metric benefits record linkage when the fields can well match. Some distance-based methods which have also been proposed for for multi-field matching do not employ machine learning algorithms\cite{guha2004merging}\cite{ananthakrishna2002eliminating}, but the distance measures are also based on schema matching.

Some other work focuses on postprocessing. The main idea is that the matching relations between records or fields should be consistent\cite{bansal2004correlation}\cite{pasula2002identity}. Such methods based on graph model can play an important role in instance matching problems as the linked data settings are more structured. They are complementary to our proposed work which uses only the literal information.

\subsection{Instance Matching}

Some of the existing links in LOD are discovered manually with some tools. One well-known instance matching tool is the Silk Link Discovery Framework\cite{volz2009discovering} which allows setting link specifications for given data sources. Domain related knowledge is required to design such specifications.

The automatic instance matching approaches are often domain specific or property matching dependent. The approach proposed in \cite{sleeman2010machine} matches the FOAF instances using SVM. Since all of the instances are described with FOAF, the features for classification are easily determined according to the limited number of properties. More domain-specific approaches can be found in \cite{raimond2008automatic} and \cite{sleeman2010computing}. Among the domain-independent approaches, \cite{hogan2007performing} matches instances according to their inverse functional properties (IFP). Such properties are not sufficient in LOD, so \cite{hogan2010some} tries to find more IFPs with a statistical method. ObjectCoref\cite{hu2011self} employs a self-learning framework to iteratively find the discriminative property-value pairs for instance matching, which are lax IFPs. RAVEN\cite{ngomo2011raven} applies active learning techniques for instance matching. Both ObjectCoref and RAVEN match the properties from different data sources by measuring value similarities. Similar ideas are proposed in the domain of \textit{schema matching}\cite{rahm2001survey}.

Finally, some papers focus on improving the efficiency of instance matching. \cite{ngomo2011limes} limits the number of candidate instance pairs to match based on the triangle equation. \cite{song2011automatically}, \cite{niu2011zhishi} and \cite{isele2011efficient} generate candidates by indexing some key words of instances. This kind of method can be applied to optimize ours.
